% Created 2012-12-16 Sun 19:43
\documentclass[11pt]{article}
\usepackage[utf8]{inputenc}
\usepackage[T1]{fontenc}
\usepackage{fixltx2e}
\usepackage{graphicx}
\usepackage{longtable}
\usepackage{float}
\usepackage{wrapfig}
\usepackage{soul}
\usepackage{textcomp}
\usepackage{marvosym}
\usepackage{wasysym}
\usepackage{latexsym}
\usepackage{amssymb}
\usepackage{hyperref}
\tolerance=1000
\usepackage{ctex}
\usepackage{mathtools}
\everymath{\displaystyle}
\usepackage{libertineotf}
\usepackage{tikz}
\usepackage{algorithm}
\usepackage{algorithmic}
\usepackage{geometry}
\geometry{left=2.5cm, right=2.5cm, top=3cm, bottom=3cm}
\providecommand{\alert}[1]{\textbf{#1}}

\title{\emph{CodeForces solution}}
\author{\emph{罗雨屏}}
\date{\today}
\hypersetup{
  pdfkeywords={},
  pdfsubject={},
  pdfcreator={Emacs Org-mode version 7.8.11}}

\begin{document}

\maketitle

\setcounter{tocdepth}{2}
\tableofcontents
\vspace*{1cm}

\newcommand{\abs}[1]{{\left\vert #1 \right\vert}}
\newcommand{\and}{\hspace{0.1cm} \textbf{and} \hspace{0.1cm}}
\newcommand{\or}{\hspace{0.1cm} \textbf{or} \hspace{0.1cm}}
\newcommand{\xor}{\hspace{0.1cm} \textbf{xor} \hspace{0.1cm}}
\newcommand{\floor}[ 1]{{\lfloor #1 \rfloor}}
\newcommand{\ceil}[ 1]{{\lceil #1 \rceil}}
\newtheorem{improve}{优化}
\newtheorem{theorem}{定理}
\newtheorem{proof}{证明}
\newtheorem{problem}{问题}
\newtheorem{definition}{定义}

\section{Volume II}
\label{sec-1}
\subsection{30E   Tricky and Clever Password}
\label{sec-1-1}
\subsubsection{题意}
\label{sec-1-1-1}

    给定一个字符串 $S$ ,要求表示成 $A + prefix + B + middle + C + suffix$ 的形式,其中除 middle 外的其余字符串均可以为空,且 prefix 和 suffix 对称相同, middle 为回文串,且 $\abs{prefix + middle + suffix}$ 最大。

    $\abs{S} \leq 10^5$
\subsubsection{算法}
\label{sec-1-1-2}

    不妨令 $T$ 为 $S$ 翻转后的串。可以知道 prefix 一定是 $T$ 的前缀。我们先用 Manacher 算法用 $O(n)$ 的时间处理出以 $i$ 为中心的最长回文串的长度。考虑枚举 middle 的终点。显然 middle 应该越长越好。那么 prefix 的最长的长度为 middle 之前串中,最长的一个子串,使得这个子串是 $T$ 的前缀,且不能超过 middle 后串的长度。我们用 KMP 算法可以求出 $S$ 的每个前缀 $pre_i$ 的最长后缀,使得这个后缀是 $T$ 的子串。枚举了 middle 的长度后, prefix 的长度显然就是前若干个前缀中最长后缀的最大值,直接求一个前缀最小值即可。

    复杂度 $O(N)$ 。

\end{document}